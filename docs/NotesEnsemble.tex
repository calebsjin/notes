\PassOptionsToPackage{unicode=true}{hyperref} % options for packages loaded elsewhere
\PassOptionsToPackage{hyphens}{url}
%
\documentclass[
]{book}
\usepackage{lmodern}
\usepackage{amssymb,amsmath}
\usepackage{ifxetex,ifluatex}
\ifnum 0\ifxetex 1\fi\ifluatex 1\fi=0 % if pdftex
  \usepackage[T1]{fontenc}
  \usepackage[utf8]{inputenc}
  \usepackage{textcomp} % provides euro and other symbols
\else % if luatex or xelatex
  \usepackage{unicode-math}
  \defaultfontfeatures{Scale=MatchLowercase}
  \defaultfontfeatures[\rmfamily]{Ligatures=TeX,Scale=1}
\fi
% use upquote if available, for straight quotes in verbatim environments
\IfFileExists{upquote.sty}{\usepackage{upquote}}{}
\IfFileExists{microtype.sty}{% use microtype if available
  \usepackage[]{microtype}
  \UseMicrotypeSet[protrusion]{basicmath} % disable protrusion for tt fonts
}{}
\makeatletter
\@ifundefined{KOMAClassName}{% if non-KOMA class
  \IfFileExists{parskip.sty}{%
    \usepackage{parskip}
  }{% else
    \setlength{\parindent}{0pt}
    \setlength{\parskip}{6pt plus 2pt minus 1pt}}
}{% if KOMA class
  \KOMAoptions{parskip=half}}
\makeatother
\usepackage{xcolor}
\IfFileExists{xurl.sty}{\usepackage{xurl}}{} % add URL line breaks if available
\IfFileExists{bookmark.sty}{\usepackage{bookmark}}{\usepackage{hyperref}}
\hypersetup{
  pdftitle={Notebook\textbar{}笔记本},
  pdfauthor={Caleb Jin\textbar{}金时强},
  pdfborder={0 0 0},
  breaklinks=true}
\urlstyle{same}  % don't use monospace font for urls
\usepackage{longtable,booktabs}
% Allow footnotes in longtable head/foot
\IfFileExists{footnotehyper.sty}{\usepackage{footnotehyper}}{\usepackage{footnote}}
\makesavenoteenv{longtable}
\usepackage{graphicx,grffile}
\makeatletter
\def\maxwidth{\ifdim\Gin@nat@width>\linewidth\linewidth\else\Gin@nat@width\fi}
\def\maxheight{\ifdim\Gin@nat@height>\textheight\textheight\else\Gin@nat@height\fi}
\makeatother
% Scale images if necessary, so that they will not overflow the page
% margins by default, and it is still possible to overwrite the defaults
% using explicit options in \includegraphics[width, height, ...]{}
\setkeys{Gin}{width=\maxwidth,height=\maxheight,keepaspectratio}
\setlength{\emergencystretch}{3em}  % prevent overfull lines
\providecommand{\tightlist}{%
  \setlength{\itemsep}{0pt}\setlength{\parskip}{0pt}}
\setcounter{secnumdepth}{5}
% Redefines (sub)paragraphs to behave more like sections
\ifx\paragraph\undefined\else
  \let\oldparagraph\paragraph
  \renewcommand{\paragraph}[1]{\oldparagraph{#1}\mbox{}}
\fi
\ifx\subparagraph\undefined\else
  \let\oldsubparagraph\subparagraph
  \renewcommand{\subparagraph}[1]{\oldsubparagraph{#1}\mbox{}}
\fi

% set default figure placement to htbp
\makeatletter
\def\fps@figure{htbp}
\makeatother

\usepackage{booktabs}
\usepackage[]{natbib}
\bibliographystyle{apalike}

\title{Notebook\textbar{}笔记本}
\author{Caleb Jin\textbar{}金时强}
\date{Last updated in 2020-01-01}

\begin{document}
\maketitle

{
\setcounter{tocdepth}{1}
\tableofcontents
}
\hypertarget{prologue}{%
\chapter{Prologue}\label{prologue}}

This is a notebook written in \textbf{bookdown}. I aggregte all notes of statistics, English study, recipe, coding, etc. to this book.

\newcommand{\ua}{{\bf a}} 
\newcommand{\uA}{{\bf A}}
\newcommand{\ub}{{\bf b}} 
\newcommand{\uB}{{\bf B}}
\newcommand{\uc}{{\bf c}}
\newcommand{\uC}{{\bf C}}
\newcommand{\ud}{{\bf d}} 
\newcommand{\uD}{{\bf D}}
\newcommand{\ue}{{\bf e}}
\newcommand{\uE}{{\bf E}}
\newcommand{\uf}{{\bf f}}
\newcommand{\uF}{{\bf F}}
\newcommand{\ug}{{\bf g}}
\newcommand{\uG}{{\bf G}}
\newcommand{\uh}{{\bf h}}
\newcommand{\uH}{{\bf H}} 
\newcommand{\ui}{{\bf i}}
\newcommand{\uI}{{\bf I}} 
\newcommand{\uj}{{\bf j}}
\newcommand{\uJ}{{\bf J}}
\newcommand{\uk}{{\bf k}}
\newcommand{\uK}{{\bf K}}
\newcommand{\ul}{{\bf l}}
\newcommand{\uL}{{\bf L}}
\newcommand{\um}{{\bf m}} 
\newcommand{\uM}{{\bf M}}
\newcommand{\un}{{\bf n}}
\newcommand{\uN}{{\bf N}}
\newcommand{\uo}{{\bf o}}
\newcommand{\uO}{{\bf O}}
\newcommand{\up}{{\bf p}}
\newcommand{\uP}{{\bf P}}
\newcommand{\uq}{{\bf q}}
\newcommand{\uQ}{{\bf Q}}
\newcommand{\ur}{{\bf r}}
\newcommand{\uR}{{\bf R}}
\newcommand{\us}{{\bf s}}
\newcommand{\uS}{{\bf S}}
\newcommand{\ut}{{\bf t}}
\newcommand{\uT}{{\bf T}}
\newcommand{\uu}{{\bf u}}
\newcommand{\uU}{{\bf U}}
\newcommand{\uv}{{\bf v}}
\newcommand{\uV}{{\bf V}}
\newcommand{\uw}{{\bf w}}
\newcommand{\uW}{{\bf W}}
\newcommand{\ux}{{\bf x}}
\newcommand{\uX}{{\bf X}} 
\newcommand{\uy}{{\bf y}} 
\newcommand{\uY}{{\bf Y}}
\newcommand{\uz}{{\bf z}}
\newcommand{\uZ}{{\bf Z}}

\newcommand\ualpha{{\boldsymbol \alpha}}
\newcommand{\ubeta}{{\boldsymbol \beta}} 
\newcommand{\bg}{{\boldsymbol \gamma}}
\newcommand{\bG}{{\boldsymbol \Gamma}}
\newcommand{\udelta}{{\boldsymbol \delta}}
\newcommand{\uDelta}{{\boldsymbol \Delta}}
\newcommand{\uepsilon}{{\boldsymbol \epsilon}}
\newcommand{\uvarepsilon}{{\boldsymbol \varepsilon}}
\newcommand{\uzeta}{{\boldsymbol \zeta}}
\newcommand{\ueta}{{\boldsymbol \eta}} 
\newcommand{\utheta}{{\boldsymbol \theta}}
\newcommand{\uvartheta}{{\boldsymbol \vartheta}}
\newcommand{\uTheta}{{\boldsymbol \Theta}}
\newcommand{\uiota}{{\boldsymbol \iota}}
\newcommand{\ukappa}{{\boldsymbol \kappa}}
\newcommand{\ulambda}{{\boldsymbol \lambda}}
\newcommand{\uLambda}{{\boldsymbol \Lambda}}
\newcommand{\umu}{{\boldsymbol \mu}} 
\newcommand{\unu}{{\boldsymbol \nu}}
\newcommand{\uxi}{{\boldsymbol \xi}}
\newcommand{\uXi}{{\boldsymbol \Xi}}
\newcommand{\uomicron}{{\boldsymbol \omicron}}
\newcommand{\uOmicron}{{\boldsymbol \Omicron}}
\newcommand{\upi}{{\boldsymbol \pi}}
\newcommand{\uPi}{{\boldsymbol \Pi}}
\newcommand{\urho}{{\boldsymbol \rho}}
\newcommand{\uvarrho}{{\boldsymbol \varrho}}
\newcommand{\usigma}{{\boldsymbol \sigma}}
\newcommand{\uSigma}{{\boldsymbol \Sigma}}
\newcommand{\utau}{{\boldsymbol \tau}}
\newcommand{\uupsilon}{{\boldsymbol \upsilon}}
\newcommand{\uUpsilon}{{\boldsymbol \Upsilon}}
\newcommand{\uphi}{{\boldsymbol \phi}}
\newcommand{\uvarphi}{{\boldsymbol \varphi}}
\newcommand{\uPhi}{{\boldsymbol \Phi}}
\newcommand{\uchi}{{\boldsymbol \chi}}
\newcommand{\upsi}{{\boldsymbol \psi}}
\newcommand{\uPsi}{{\boldsymbol \Psi}}
\newcommand{\uomega}{{\boldsymbol\omega}}
\newcommand{\uOmega}{{\boldsymbol\Omega}}

\newcommand{\0}{{\boldsymbol 0}} 
\newcommand{\1}{{\boldsymbol 1}} 
\newcommand{\T}{{ \top}} 
\newcommand{\diag}{{\rm diag}}

\newcommand\nbd{{\rm nbd}}

\hypertarget{stats}{%
\chapter{Statistics}\label{stats}}

\hypertarget{bayesian-linear-model}{%
\section{Bayesian Linear Model}\label{bayesian-linear-model}}

\hypertarget{project-1}{%
\section{Project 1}\label{project-1}}

\hypertarget{project-2}{%
\section{Project 2}\label{project-2}}

\hypertarget{project-3}{%
\section{Project 3}\label{project-3}}

\[\hat{\boldsymbol \beta}_{{\boldsymbol \gamma}^+} = ({\bf X}_{{\boldsymbol \gamma}^+}^{{ \top}}{\bf X}_{{\boldsymbol \gamma}^+}+\lambda^{-1}{\bf I}_{|{{\boldsymbol \gamma}^+}|})^{-1}{\bf X}_{{\boldsymbol \gamma}^+}^{{ \top}}{\bf y}\]

\hypertarget{coding}{%
\chapter{Coding}\label{coding}}

\hypertarget{python}{%
\section{Python}\label{python}}

\hypertarget{mysql}{%
\section{MySQL}\label{mysql}}

\hypertarget{r}{%
\section{R}\label{r}}

\hypertarget{english}{%
\chapter{English}\label{english}}

We describe our methods in this chapter.

\hypertarget{recipe}{%
\chapter{Recipe}\label{recipe}}

Some \emph{significant} applications are demonstrated in this chapter.

\hypertarget{example-one}{%
\section{Example one}\label{example-one}}

\hypertarget{example-two}{%
\section{Example two}\label{example-two}}

\hypertarget{final-words}{%
\chapter{Final Words}\label{final-words}}

We have finished a nice book.

  \bibliography{book.bib,packages.bib}

\end{document}
